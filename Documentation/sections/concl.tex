\section{Conclusiones}
En general estamos muy satisfechos con la práctica. Machine learing es un tema que nos interesa mucho a los 3, y consideramos que los ejercicios propuestos nos han ayudado a ampliar nuestro conocimiento en este campo.

Con el objetivo de conseguir mejores resultados, hemos hecho diferentes iteraciones, probado diferentes tecnologías y realizado numerosos experimentos. Inicialmente desarrollamos el perceptrón simple y la red neuronal, en esta primera aproximación pudimos obtener los primeros resultados a partir de los inputs obtenidos en cada juego. 

Tras haber realizado esta primera aproximación y viendo que teníamos una red que funcionaba y tiempo de sobra, nos animamos a implementar el algoritmo genético sobre la red neuronal, tratar los datos para 4 juegos distintos y encontrar topologías y conjuntos de entrenamiento que funcionaran bien era una tarea costosa, el algoritmo genético resolvía en parte este problema.

Por otro lado el genético añade una complejidad extra como es encontrar una función de fitness ideal para cada juego, cuya dificultad variará en función la complejidad del juego. Sin embargo, esta tarea es menos tediosa que sanitizar y tratar todos los datos del juego.

Además, realizamos un intento de implementación de \ac{neat} que no se finalizó debido a la falta de tiempo y complejidad de la tarea. A pesar de esto, nos parece que es la tecnología más interesante de todas y no descartamos finalizarla en un futuro.

En conclusión,el algoritmo genético ha sido el que finalmente hemos utilizado para resolver los problemas propuestos debido a que nos ha parecido una tecnología mucho más interesante y que se ajusta más a este tipo de problemas.